\input{text/diss}
\usepackage{gensymb}
\usepackage{textcomp}
\usepackage{pythontex}

\begin{document}
\def\labauthors{Карусевич А.А, Понур К.А.}
\def\labgroup{430}
\def\department{Кафедра электродинамики}
\def\labnumber{1}
\def\labtheme{Исследование отражательного клистрона}

\renewcommand{\Re}{\operatorname{Re}}
\renewcommand{\Im}{\operatorname{Im}}
\renewcommand{\phi}{\varphi}
\renewcommand{\hat}{\widehat}

\input{text/titlepage}
\tableofcontents
\newpage

\section{Теоретическая часть}
\subsection{Введение}
\subsection{Резонатор клистрона}
\subsection{Модуляция электронов в пучке}
\subsection{Модуляция электронного потока по плотности}
\subsection{Возбуждение резонатора клистрона током пучка}

\section{Экспериментальная часть}

\end{document}